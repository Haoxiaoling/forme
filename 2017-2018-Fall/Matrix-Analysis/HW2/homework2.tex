% good luck;)
\documentclass[a4paper]{article}
\usepackage{xeCJK}
\usepackage{geometry}
\usepackage[colorlinks,linkcolor=blue]{hyperref}
\newcommand\email[1]{\href{mailto:#1}{\uline{\nolinkurl{#1}}}}
\geometry{left=2.5cm,right=2.5cm,top=2.5cm,bottom=4cm}
\title{Matrix Analysis Homework 2}
\author{龙肖灵 \\Xiaoling Long\\Student ID.:81943968\\Email:\email{longxl@shanghaitech.edu.cn}}
\usepackage{graphicx}
\usepackage{amsmath, amsthm, amssymb}
\usepackage{subfloat}
\newtheorem{prop}{Proposition}
\usepackage{ulem}
\usepackage{indentfirst}
\begin{document}
\maketitle

\begin{description}
\item[Problem 1](4/10, \textit{If $G$ is an abelian group, then $End(G)$ is a ring}):

Let $G$ be an abelian group with group operation $*$. For $\phi,\psi \in End(G)$,
consider the definition of $\phi +\psi$ and $\phi\psi$ that we gave in the class.
Verify that $End(G)$ is a ring under these two operations, i.e.,
check that all the axioms of the ring structure are true. For example,
verify that for $\phi,\psi,\chi \in End(G)$, we have that $\phi(\psi+\chi)=\phi\psi+\phi\chi$.
\begin{proof}
      $$\forall \phi,\psi,\chi \in End(G)$$

      \begin{enumerate}
        \item       We have that, \begin{align*}
            [(\phi+\psi)+\chi](g)&=[\phi+\psi](g)*\chi(g)\\
            &=[\phi(g)*\psi(g)]*\chi(g)
            \intertext{G is an abelian group.}
           [\phi(g)*\psi(g)]*\chi(g)&=\phi(g)*[\psi(g)*\chi(g)]\\
            &=[\phi+(\psi+\chi)](g)
          \end{align*}
          So(+ is associative), $$(\phi+\psi)+\chi=\phi+(\psi+\chi)$$
          \item And then,
          \begin{align*}
            [\phi+\psi](g)&=\phi(g)*\psi(g)\\
            &=\psi(g)*\phi(g)\\
            &=[\psi+\phi](g)\quad\text{(+ is commutative)}
          \end{align*}
          \item Identity under $+$,\\
          Let $\forall g \in End(V)\quad 0_{End(V)}: g \mapsto e_{G}\text{($e_{G}$ is identity of group $G$)} $
          be identity of group $End(V)$ under $+$.
          \begin{align*}
            [\phi+0_{End(V)}](g)&=\phi(g)*0_{End(V)}(g)\\
            &=\phi(g)*e_{G}(g)\\
            &=\phi(g)
          \end{align*}
          We can say that, $$\phi+0_{End(g)} =\phi$$
          So $0_{End(V)}$ is identity of $End(V)$ under operation $+$
          \item Inverse of $+$,let $-\phi: g\mapsto (\phi(g))^{-1}$.\\
          We can know,
          \begin{align*}
            [\phi+(-\phi)](g)&=\phi(g)*([-\phi])(g)\\
            &=\phi(g)*(\phi(g))^{-1}\\
            &=e_{G}\\
            &=0_{End(V)}(g)
          \end{align*}
          So, we can say that $$[\phi+(-\phi)]=0_{End(V)}$$
          \item About $\cdot$, we have that,
          \begin{align*}
            [(\phi\cdot\psi)\cdot\chi](g)&=[\phi\cdot\psi](\chi(g))\\
            &=\phi[\psi(\chi(g))]
            \intertext{Also, we have this,}
            [\phi\cdot(\psi\cdot\chi)](g)&=\phi[(\psi\cdot\chi)(g)]\\
            &=\phi[\psi(\chi(g))]
            \intertext{Finally, $\cdot$ is associative.}
            [(\phi\cdot\psi)\cdot\chi](g)&=[\phi\cdot(\psi\cdot\chi)](g)
          \end{align*}
          \item Let $\forall g \in G \quad 1_{End(V)}: g \mapsto g$, then,
          \begin{align*}
            [\phi \cdot 1_{End(V)}](g)&=\phi(1_{End(V)}(g))\\
            &=\phi(g)
            \intertext{Finally, we get}
            (\phi \cdot 1_{End(V)})&=\phi
          \end{align*}
          \item Left distributivity.
          \begin{align*}
            [\phi(\psi+\chi)](g)&=\phi[\psi(g)*\chi(g)]\\
            &=[\phi(\psi(g))]*[\phi(\chi(g))]\\
            \intertext{And}
            [\phi\psi+\phi\chi](g)&=[\phi\psi(g)]*[\phi\psi(g)]\\
            &=[\phi(\psi(g))]*[\phi(\chi(g))]\\
            &=[\phi(\psi+\chi)](g)\quad\text{(Done.)}
          \end{align*}
          \item Right distributivity.
          \begin{align*}
            [(\phi+\psi)\chi](g)&=[\phi+\psi](\chi(g))\\
            &=[\phi(\chi(g))]*[\psi(\chi(g))]\\
            &=[\phi\chi](g)*[\psi\chi](g)\\
            &=[\phi\chi+\psi\chi](g)\quad\text{(Done.)}
          \end{align*}
      \end{enumerate}
      So, $End(V)$ satisfies the all properties of ring.$End(V)$ is a ring under $+$ and $\cdot $.

\end{proof}
\item[Problem 2](3/10, \textit{Injectivity of ring homomorphisms on a field}):{
    Let $V \ne 0$ be a vector space over a field $F$ and let $\sigma: F \to End(V)$ be its associated ring homomorphism. Show that if $0 \ne c \in F$, then $\sigma(c)$ can not be the zero element of $End(V)$. More generally, show that if $R\ne 0$ is any ring and $\tau: F \to R$ is any ring homomorphism, then $\tau(c) = 0 \Rightarrow c = 0$.
}

\begin{proof}
If $\exists c \in F$ and $c\ne 0$, $\sigma(c)=0_{End(V)}$, we have that,
  \begin{align*}
    \forall a \in F \quad \sigma(c\cdot a)&=\sigma(c)\cdot \sigma(a)\\
    &=0_{End(V)}\cdot \sigma(a)\\
    &=0_{End(V)}
    \intertext{Actually, we have }
    \forall f & \in F \quad \exists a \in F \quad f=c\cdot a\\
    \to \forall & f \in  F \quad \sigma(f)=0_{End(V)}
  \end{align*}
  So, $End(V)$ is a zero ring. And if $End(V)$ is not a zero ring, we have $\forall f\in F :
  \sigma(f) \ne 0_{End(V)}$, except $c =0$.Now I will show that.\\
  Let $0$ be identity element of $F$. And we can say that,$$\forall c \in F \quad c+0=0+c=c$$
  Let $c\ne 0$ ,so,
  \begin{align*}
    \sigma(c+0)&=\sigma(c)+\sigma(0)\\
    \sigma(c)&=\sigma(c)+\sigma(0)\\
    \intertext{Because}
    \forall c \in F : \sigma(c)+0_{End(V)}&=0_{End(V)}+\sigma(c)=\sigma(c)\\
    \sigma(0)&=0_{End(V)}
  \end{align*}
  So, we can give a conclusion that if $End(V)$ is not a zero ring, $\forall 0\ne c \in F \to
  \sigma(c)$ can not be the zero element of $End(V)$.\\
  More generally, by contrary, supposed $\exists c \in R$ and $c\ne 0$ let $\tau(c)=0$
  \begin{align*}
    \forall a \in R \quad &\exists b \in R : a=b\cdot c \\
     \tau(b\cdot c)&=\tau(b)\cdot \tau(c)\\
    &=\tau(b)\cdot 0\\
    &=0\\
    \to \tau(a)&=0
  \end{align*}
  So $F$ is a zero ring. Contraty to condition. And if $a=0$ we have $\tau(0)$
  always equals to $0$.
  We can conclude that $$\tau(c)=0 \to c=0$$
   Done.
\end{proof}

\item[Problem 3](3/10, \textit{An adventure in ring theory}):{
    Let $R$ be a commutative ring with additive identity 0 and multiplicative identity 1.
    Let r be a nilpotent element of $R$, i.e., there exists a positive integer $n$ such
    that $r^{n} = 0$. Show that the element $u := r + 1$ is an unit of $R$, i.e.,
    show that there exists some element $p \in R$, such that $up = 1$.
    \emph{Terminology: the set of invertible elements of a ring is a group,
    known as the group of units of the ring. Thus the group of units of a field
    is the entire field except the zero element. }
}

\begin{proof}
  It means that there always is a inverse if $u=r+1$.
  Because the properties of the ring. We can say that.
  \begin{align}
    u \cdot 1&=u\notag\\
    &=r+1\tag{1}\\
    r+0&=r\notag\\
    \intertext{Let $n=2$, then we have }
    r^{2}&=0\tag{2}\notag\\
    u\cdot r&=(r+1)\cdot r\notag\\
    &=r^{2}+r\tag{3}
  \end{align}
  Bring (2) into (3), we can get
  \begin{align}
    u\cdot r=r\tag{4}
  \end{align}
  Then (1) minus (4),$$u\cdot 1-u\cdot r=1$$
  And $R$ is a commutative ring with multiplicative.
  \begin{align*}
    u\cdot (1-r)&=1\\
    (1-r)\cdot u&=1
  \end{align*}
  And we can easily get $1-r \in R$. Finally, we can say $u:=r+1$ is
  an unit of $R$. And the inverse of $u$ is $1-r$.\\
  Done.
\end{proof}

\end{description}
\end{document}
