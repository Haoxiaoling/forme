% good luck;)
\documentclass[a4paper]{article}
\usepackage{geometry}
\geometry{left=4cm,right=4cm,top=1cm,bottom=4cm}
\title{Homework 1}
\author{Xiaoling Long\\Student ID.:81943968\\email:longxl@shanghaitech.edu.cn}
\usepackage[utf8]{inputenc}
\usepackage{graphicx}
\usepackage[colorlinks,linkcolor=red]{hyperref}
\usepackage{amsmath, amsthm, amssymb}
\usepackage{subfloat}
\newtheorem{prop}{Proposition}
\usepackage{ulem}
\usepackage{indentfirst}
\begin{document}
\maketitle

\begin{description}
\item[Problem 1](4/10, \textit{If $G$ is an abelian group, then $End(G)$ is a ring}):

Let $G$ be an abelian group with group operation $*$. For $\phi,\psi \in End(G)$,
consider the definition of $\phi +\psi$ and $\phi\psi$ that we gave in the class.
Verify that $End(G)$ is a ring under these two operations, i.e.,
check that all the axioms of the ring structure are true. For example,
verify that for $\phi,\psi,\chi \in End(G)$, we have that $\phi(\psi+\chi)=\phi\psi+\phi\chi$.
\begin{proof}
      $$\forall \phi,\psi,\chi \in End(G)$$
      We have that,
  \begin{align*}
    [(\phi+\psi)+\chi](g)&=[\phi+\psi](g)*\chi(g)\\
    &=[\phi(g)*\psi(g)]*\chi(g)
    \intertext{G is an abelian group.}
   [\phi(g)*\psi(g)]*\chi(g)&=\phi(g)*[\psi(g)*\chi(g)]\\
    &=[\phi+(\psi+\chi)](g)
  \end{align*}
  So, $$(\phi+\psi)+\chi=\phi+(\psi+\chi)$$
\end{proof}
\item[Problem 2](4/10, \textit{Behavior of identity element and inverses under
  group homomorphisms}):

Let $G$, $H$ be groups with group operations $*_G$, $*_H$ and identity
elements $e_G$, $e_H$ respectively. Let $\phi:G\to H$ be a group
homomorphism. Show that $\phi(e_G)=e_H$ and that $\forall g\in G$ we
have that $\phi(g^{-1})=(\phi(g))^{-1}$.
\begin{proof}
Here goes your anwser.
\end{proof}
\item[Problem 3](2/10, \textit{Group isomorphism is a transitive relation}):

Let $G$, $H$, $K$ be groups such that $G$ is isomorphic to $H$, and
$H$ is isomorphic to $K$. Show that $G$ is isomorphic to $K$.
\begin{proof}
Here goes your anwser.
\end{proof}

\end{description}
\end{document}
