\documentclass[a4paper]{article}
\usepackage{xeCJK}
\usepackage{geometry}
\geometry{left=2.5cm,right=2.5cm,top=3cm,bottom=3cm}
\title{Matrix Analysis Homework 8}
\author{龙肖灵 \\Xiaoling Long\\Student ID.:81943968\\email:longxl@shanghaitech.edu.cn}
\usepackage{graphicx}
\usepackage[colorlinks,linkcolor=red]{hyperref}
\usepackage{amsmath, amsthm, amssymb}
\usepackage{mathtools}
\usepackage{subfloat}
\usepackage{stmaryrd}
\usepackage{enumerate}
\newtheorem{prop}{Proposition}
\usepackage{ulem}
\usepackage{indentfirst}
\begin{document}
\maketitle

For A symmetric $n\times n$ matrix, we assume the following ordering on its eigenvalues: $$\lambda_{1}(A)\ge \lambda_{2}(A)\ge \cdots \ge \lambda_{n}(A)$$
\begin{description}
  \item[Problem 1]: \textit{(Finish the proof of the second part of the Weyl-II theorem)}\\
  Let $A$,$B$ be $n\times n$ symmetric matrices. Prove that $\lambda_{j}(A)+\lambda_{k}(B)\le\lambda_{j+k−n}(A+B),\forall i,j=1,\cdots,n$.

  \begin{proof}\

    Done.
  \end{proof}

  \item[Problem 2]:\textit{(Finish the proof of the second part of the Interlacing-II theorem)}\\
   Let $A$ be $n\times n$ symmetric matrix. Let $B$ be an $r\times r$ principal submatrix of $A$, obtained by deleting rows/columns $i_{1},\cdots,i_{n−r}$. Show that $\lambda_{k}(B)\le\lambda_{k}(A),\forall k=1,\cdots,r$.

  \begin{proof}\
    Done.
  \end{proof}

  \item[Problem 3]:\textit{(Finish the proof of the second part of the variational characterization of sums of eigenvalues)}\\
  Let $A$ be $n\times n$ symmetric matrix. Prove that $\sum_{i=n−k+1}^{n}\lambda_{i}(A)=\min_{U\in \mathbb{R}^{n×k},U^{T}U=I_{k}}trace(U^{T}AU)$.

  \begin{proof}

    Done.
  \end{proof}

  \item[Problem 4]:\\
  Let $A$ be an $n\times n$ symmetric matrix. Prove that $\lambda_{n}(A)\le a_{ii}\le \lambda_{1}(A),\forall i=1,\cdots,n$.

  \begin{proof}\
    Done.
  \end{proof}

  \item[Problem 5]:\\
   Let $A$, $B$ be $n\times n$ symmetric matrices. Prove that $\sum^{k}_{i=1}(\lambda_{i}(A)+\lambda_{i}(B))\ge\sum_{i=1}^{k}\lambda_{i}(A+B),\ \forall k=1,\cdots,n$.\\
    Hint: Use the variational characterization of the sum of eigenvalues.

  \begin{proof}\ \\
    Done.
  \end{proof}


\end{description}

\end{document}
