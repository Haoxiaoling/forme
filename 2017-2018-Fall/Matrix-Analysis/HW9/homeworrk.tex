\documentclass[a4paper]{article}
\usepackage{xeCJK}
\usepackage{geometry}
\geometry{left=2.5cm,right=2.5cm,top=3cm,bottom=3cm}
\title{Matrix Analysis Homework 9}
\author{龙肖灵 \\Xiaoling Long\\Student ID.:81943968\\email:longxl@shanghaitech.edu.cn}
\usepackage{graphicx}
\usepackage[colorlinks,linkcolor=red]{hyperref}
\usepackage{amsmath, amsthm, amssymb}
\usepackage{mathtools}
\usepackage{subfloat}
\usepackage{stmaryrd}
\usepackage{enumerate}
\newtheorem{prop}{Proposition}
\usepackage{ulem}
\usepackage{indentfirst}
\begin{document}
\maketitle

For A symmetric $n\times n$ matrix, we assume the following ordering on its eigenvalues: $$\lambda_{1}(A)\ge \lambda_{2}(A)\ge \cdots \ge \lambda_{n}(A)$$
\begin{description}
  \item[Problem 1]: \textit{(Finish the proof of the second part of the Weyl-II theorem)}\\
  Let $A$,$B$ be $n\times n$ symmetric matrices. Prove that $\lambda_{j}(A)+\lambda_{k}(B)\le\lambda_{j+k−n}(A+B),\forall j,k=1,\cdots,n$.

  \begin{proof}\ \\
    We already know that $\lambda_{j+k-1}\le \lambda_{j}(A)+\lambda_{k}(B)$.
    \begin{align*}
      -\lambda_{n-(j+k-1)+1}(A+B)&= \lambda_{j+k-1}(-A-B)\\
      -\lambda_{n-j-k+2}(A+B)&\le \lambda_{j}(-A)+\lambda_{k}(-B)\\
      &=-\lambda_{n-j+1}(A)-\lambda_{n-k+1}(B)
    \end{align*}
    And let $j^{\prime}=n-j+1$, $k^{\prime}=n-k+1$, we can get that $$n-j-k+2=j^{\prime}+k^{\prime}-n$$
    Finally, we can get that $$\lambda_{j^{\prime}+k^{\prime}}(A+B)\ge \lambda_{j^{\prime}}(A)+\lambda_{k^{\prime}}(B),\ \forall j^{\prime},k^{\prime}=1,\cdots,n$$\
    Done.
  \end{proof}

  \item[Problem 2]:\textit{(Finish the proof of the second part of the Interlacing-II theorem)}\\
   Let $A$ be $n\times n$ symmetric matrix. Let $B$ be an $r\times r$ principal submatrix of $A$, obtained by deleting rows/columns $i_{1},\cdots,i_{n−r}$. Show that $\lambda_{k}(B)\le\lambda_{k}(A),\forall k=1,\cdots,r$.

  \begin{proof}\ \\
    Similarly, we already have that $\lambda_{n+k-r}(A)\le \lambda_{k}(B)$.
    \begin{align*}
      -\lambda_{n-(n+k-r)+1}(A)&=\lambda_{n+k-r}(-A)\\
      -\lambda_{r-k+1}(A)&\le \lambda_{k}(-B)\\
      &=-\lambda_{r-k+1}(B)
    \end{align*}
    So, let $k^{\prime}=r-k+1$ and when $k=1,\cdots,r$, $k^{\prime}=r,\cdots,1$ respectively. So we can get that $$\lambda_{k^{\prime}}(B)\le\lambda_{k^{\prime}}(A),\ \forall k^{\prime}=1,\cdots,r$$
    Done.
  \end{proof}

  \item[Problem 3]:\textit{(Finish the proof of the second part of the variational characterization of sums of eigenvalues)}\\
  Let $A$ be $n\times n$ symmetric matrix. Prove that $\sum_{i=n−k+1}^{n}\lambda_{i}(A)=\min_{U\in \mathbb{R}^{n×k},U^{T}U=I_{k}}trace(U^{T}AU)$.

  \begin{proof}\ \\
    Similarly, we have that $U^{T}AU$ is part of $V^{T}AV$. Still by \bfseries Interlacing-II \mdseries, we get another inequality as follow.
    \begin{equation}
      \lambda_{i}(U^{T}AU)\ge \lambda_{n+i-k}(V^{T}AV)=\lambda_{n+i-k}(A)\label{sum}
    \end{equation}
    Sum \ref{sum} for $i=1,\cdots,k$, we can get that $$trace(U^{T}AU)=\sum\limits_{i=1}^{k}\lambda_{i}(U^{T}AU)\ge \sum\limits_{i=1}^{k}\lambda_{n+i-k}(A)=\sum\limits_{j=n-k+1}^{n}\lambda_{j}(A)$$
    Done.
  \end{proof}

  \item[Problem 4]:\\
  Let $A$ be an $n\times n$ symmetric matrix. Prove that $\lambda_{n}(A)\le a_{ii}\le \lambda_{1}(A),\forall i=1,\cdots,n$.

  \begin{proof}\ \\
    Let $U=e_{i},\ n=1,\cdots,n$, so $dim(U)=n\times 1$ and $U^{T}U=1$. By \textit{Interlacing-II}, we have $$\sum\limits_{i=n-k+1}^{n}\lambda_{i}(A)\le trace(U^{T}AU)\le \sum\limits_{i=1}^{k}\lambda_{i}(A)$$
    Now $k=1$, and $U^{T}AU=a_{ii}$, Finally, we get that, $\lambda_{n}(A)\le a_{ii}\le \lambda_{1}(A),\ i=1,\cdots,n$.\\
    Done.
  \end{proof}

  \item[Problem 5]:\\
   Let $A$, $B$ be $n\times n$ symmetric matrices. Prove that $\sum^{k}_{i=1}(\lambda_{i}(A)+\lambda_{i}(B))\ge\sum_{i=1}^{k}\lambda_{i}(A+B),\ \forall k=1,\cdots,n$.\\
    Hint: Use the variational characterization of the sum of eigenvalues.

  \begin{proof}\ \\
    By \textit{the variational characterization of the sum of eigenvalues}.
    \begin{align*}
      \sum_{i=1}^{k}\lambda_{i}(A+B)&=\max\limits_{U: n\times k \atop U^{T}U=I_{k}}(U^{T}(A+B)U)\\
      &=\max\limits_{U: n\times k\atop U^{T}U=I_{k}}(U^{T}AU+U^{T}BU)\\
      &\le \max\limits_{U: n\times k\atop U^{T}U=I_{k}}(U^{T}AU)+\max\limits_{U: n\times k\atop U^{T}U=I_{k}}(U^{T}BU)\\
      &=\sum_{i=1}^{k}\lambda_{i}(A)+\sum_{i=1}^{k}\lambda_{i}(B)
    \end{align*}
    Finally, we get that $\sum_{i=1}^{k}(A+B)\le \sum_{i=1}^{k}\lambda_{i}(A)+\sum_{i=1}^{k}(B)$.\\
    Done.
  \end{proof}


\end{description}

\end{document}
