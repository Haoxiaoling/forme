\documentclass[a4paper]{article}
\usepackage{xeCJK}
\usepackage{geometry}
\geometry{left=2.5cm,right=2.5cm,top=3cm,bottom=3cm}
\title{Matrix Analysis Homework 4}
\author{龙肖灵 \\Xiaoling Long\\Student ID.:81943968\\email:longxl@shanghaitech.edu.cn}
\usepackage{graphicx}
\usepackage{amsmath, amsthm, amssymb}
\usepackage{subfloat}
\usepackage{enumerate}
\newtheorem{prop}{Proposition}
\usepackage{ulem}
\usepackage{indentfirst}
\begin{document}
\maketitle

\begin{description}
\item[Problem 1](\textit{Theorem 2.1.1 in Roman}):\\
Prove that the set $\mathcal{L}(\mathcal{V},\mathcal{W})$ of all linear transformations from vector space $\mathcal{V}$ to a vector space $\mathcal{W}$ is itself a vector space.

\begin{proof}
  We should show that this set fit all properties of a vector space.Let $v \in \mathcal{V} , w \in \mathcal{W} $.
  \begin{enumerate}[ 1) ]
    \item \textbf{Closure of addition}\ $\forall \tau,\sigma \in \mathcal{L}\left(\mathcal{V},\mathcal{W}\right)$.
    We can get that,
      $$[\tau+\sigma](v)=\tau(v)+\sigma(v) \in \mathcal{W}$$
      So, $\mathcal{L}\left(\mathcal{V},\mathcal{W}\right)$ is closure of addition.
    \item \textbf{Commutativity of addition}\ $\forall \tau,\sigma \in \mathcal{L}\left(\mathcal{V},\mathcal{W}\right)$.
    \begin{align*}
      [\tau+\sigma](v)&=\tau(v)+\sigma(v)
      &=\sigma(v)+\tau(v)
      &=[\sigma+\tau](v)
    \end{align*}
    So, we can say that,$$\tau+\sigma=\sigma+\tau$$
    \item \textbf{Associativity of additon}\  $\forall \tau,\sigma,\phi \in \mathcal{L}\left(\mathcal{V},\mathcal{W}\right)$. We have that,
    \begin{align*}
      [\tau+(\sigma+\phi)](v)&=\tau(v)+[\sigma+\phi](v)\\
      &=\tau(v)+\sigma(v)+\phi(v) \in \mathcal{W}\\
      &=[\tau+\sigma](v)+\phi(v)\\
      &=[(\tau+\sigma)+\phi](v)
    \end{align*}
    Finally, we get $$\tau+(\sigma+\phi)=(\tau+\sigma)+\phi$$

    \item \textbf{Existence of zero}\ Let $0_{\mathcal{L}\left(\mathcal{V},\mathcal{W}\right)}$ maps all $v\in \mathcal{V}$ to $0\in \mathcal{W}$. Then, $\forall \tau \in \mathcal{L}\left(\mathcal{V},\mathcal{W}\right)$,
    $$[\tau+0_{\mathcal{L}\left(\mathcal{V},\mathcal{W}\right)}](v)=\tau(v)+0_{\mathcal{L}\left(\mathcal{V},\mathcal{W}\right)}(v)=\tau(v)$$
    \item \textbf{Existence of additive inverses}\ $\forall \tau \in \mathcal{L}\left(\mathcal{V},\mathcal{W}\right)$.
    $$[\tau+(-\tau)](v)=\tau(v)+(-\tau(v))=0=0_{\mathcal{L}\left(\mathcal{V},\mathcal{W}\right)}(v)$$
    \item \textbf{Scalar multiplication}\ For all scalars $a,b\in \mathcal{F}$ and for all linear transformation $\tau,\sigma \in \mathcal{L}\left(\mathcal{V},\mathcal{W}\right)$,
    \begin{align*}
      a[\tau+\sigma](v)&=a[\tau(v)+\sigma(v)]\\
      &=a\tau(v)=a\sigma(v)\\
      &=[a\tau+a\sigma](v)\\
      \Rightarrow a(\tau+\sigma)&=a\tau+a\sigma\\
      [(a+b)\tau](v)&=(a+b)\tau(v)\\
      &=a\tau(v)+b\tau(v)\\
      &=[a\tau+b\tau](v)\\
      \Rightarrow (a+b)\tau&=a\tau+b\tau\\
      [(ab)\tau](v)&=(ab)[\tau(v)]\\
      &=a[b\tau(v)]\\
      \Rightarrow (ab)\tau&=a(b\tau)
    \end{align*}
  \end{enumerate}
  All properities of vector space fit for the set  $\mathcal{L}(\mathcal{V},\mathcal{W})$. So, it a a vector space.
\end{proof}

\item[Problem 2](\textit{Theorem 2.5 in Roman}):\\
Prove that a linear transformation $\tau \in \mathcal{L}(\mathcal{V},\mathcal{W})$ is an isomorphism if and only if there is a basis $\mathfrak{B}_\mathcal{V}$ for $\mathcal{V}$ for which $\tau\mathfrak{B}_\mathcal{V}$ is a basis for $\mathcal{W}$. Prove that in this case, $\tau$ maps any basis of $\mathcal{V}$ to a basis of $\mathcal{W}$.

\begin{proof}
First of all, sufficiency of this.\\
$\forall v \in \mathcal{V}$ has an unique linear combination of the vectors in $\mathfrak{B}_{\mathcal{V}}$. Let $\mathfrak{B}_{\mathcal{V}}=\{v_{i}\  |\ i\in I \}$, we have that,
\begin{equation*}
  v=a_{1}v_{1}+a_{2}v_{2}+\cdots +a_{n}v_{n}
\end{equation*}
Then we have that,
\begin{equation*}
  \tau(v)=a_{1}\tau(v_{1})+a_{2}\tau(v_{2})+\cdots +a_{n}\tau(v_{n})
\end{equation*}
$\tau{\mathfrak{B}_{\mathcal{V}}}$ is a basis for $\mathcal{W}$. So $\forall v \in \mathcal{V}, \exists!\  w \in \mathcal{W}, \text{s.t.}\ \tau(v)=w$. So $\tau$ is injective.
And $\forall w \in \mathcal{W}$ also has an unique linear combination of the vectors in $\tau(\mathfrak{B}_{\mathcal{V}})$. So all vectors in $\mathcal{W}$ have $w=\tau(v)$. So $\tau$
is surjective. So $\tau$ is an isomorphism.\\
For necessity of this. $\tau$ is isomorphism means $\tau$ is injective and surjective. So, ker($\tau$)$=\{0\}$, and there doesn't exist a linear combination whose coefficients aren't all $0$.
For all $w\in \mathcal{W}$, we have an unique $v\in \mathcal{V}$, $w=\tau(v)$. So, $$w=\tau(v)=\tau(a_{1}v_{1}+a_{2}v_{2}+\cdots +a_{n}v_{n})=a_{1}\tau(v_{1})+a_{2}\tau(v_{2})+\cdots +a_{n}\tau(v_{n})$$
Only when all coefficients are $0$, $v=0$ and $w=\tau(v)=0$ . So $\tau(\mathfrak{B}_{\mathcal{V}})$ is linearly independent. And all $w$ has a linear combination of $\tau(v_{i})\in \tau(\mathfrak{B}_{\mathcal{V}})$, so it is a spanning set for $\mathcal{W}$. So $\tau\mathfrak{B}_\mathcal{V}$ is a basis for $\mathcal{W}$.

All basis $\mathfrak{B}$ of $\mathcal{V}$ has simillar properities. Since $\tau$ is isomorphism, $\tau$ is injective and surjective. For all $w=\tau(v)$ can write as an unique linear combination with coefficients. And only when all coefficients are $0$, $w=0$. So $\tau(\mathfrak{B})$ is linearly independent and is a spanning set. Hence, $\tau$ maps any basis of $\mathcal{V}$ to a basis of $\mathcal{W}$.
\end{proof}

\item[Problem 3](\textit{Corollary 2.9 in Roman}):\\
Let $\tau \in \mathcal{L}(\mathcal{V},\mathcal{W})$ be a linear transformation from vector space $\mathcal{V}$ to vector space $\mathcal{W}$, where $\mathcal{V},\mathcal{W}$ are both finite dimensional vector spaces with $dim(\mathcal{V})=dim(\mathcal{W})$. Prove that $\tau$ is injective if and only if it is surjective.

\begin{proof}
Prove injective($\Rightarrow$). \\
$\tau$ is surjective. So, im($\tau$)$=\mathcal{W}$. From theorem 2.8, we know that
\begin{align*}
  dim\left(ker\left(\tau\right)\right)+dim\left(im\left(\tau\right)\right)&=dim\left(\mathcal{V}\right)\\
  dim\left(ker\left(\tau\right)\right)+dim\left(\mathcal{W}\right)&=dim\left(\mathcal{V}\right)
\end{align*}
And we know that $dim(\mathcal{V})=dim(\mathcal{W})$. So we can get that, $$dim\left(ker\left(\tau\right)\right)=0$$
So $ker\left(\tau\right)$ must be $0$. So $\tau$ is injective.


Prove surjective($\Leftarrow$).\\
We have an injective $\tau$. So, $ker\left(\tau\right)=\{0\} $ and $dim\left(ker\left(\tau\right)\right)=0$. Also we have $$dim\left(ker\left(\tau\right)\right)+dim\left(im\left(\tau\right)\right)=dim\left(\mathcal{V}\right)$$
So we can say $dim\left(im\left(\tau\right)\right)=dim\left(\mathcal{V}\right)=dim\left(\mathcal{W}\right)$. \\
And we know that $im\left(\tau\right)\subseteq\mathcal{W}$ $\Rightarrow$\ $im\left(\tau\right)=\mathcal{W}$.
Hence $\tau$ is surjective.\\
Done.
\end{proof}

\end{description}

\end{document}
