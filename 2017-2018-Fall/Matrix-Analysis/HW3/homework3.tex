\documentclass[a4paper]{article}
\usepackage{xeCJK}
\usepackage{geometry}
\geometry{left=2.5cm,right=2.5cm,top=3cm,bottom=3cm}
\title{Homework 3}
\author{龙肖灵 \\Xiaoling Long\\Student ID.:81943968\\email:longxl@shanghaitech.edu.cn}
\usepackage{graphicx}
\usepackage[colorlinks,linkcolor=red]{hyperref}
\usepackage{amsmath, amsthm, amssymb}
\usepackage{subfloat}
\newtheorem{prop}{Proposition}
\usepackage{ulem}
\usepackage{indentfirst}
\begin{document}
\maketitle

\begin{description}
  \item[Problem 1](3/10, Every subspace has a complement):{

  Let $\mathcal{V}$ be a vector space and $\mathcal{S}$ a subspace of $\mathcal{V}$. Show that $\mathcal{S}$ has
  a complement in $\mathcal{V}$, i.e.,that there exists a subspace $\mathcal{T}$ of $\mathcal{V}$,
  such that $\mathcal{V}=\mathcal{S}\oplus \mathcal{T}$}
  \begin{proof}
    $\mathcal{S}$ is a subspace of $\mathcal{V}$, Let $\mathcal{T}$ be a subspace of $\mathcal{V}$ and $$\mathcal{S} \cap
    \mathcal{T} =\{0\} $$
    Let $s \in \mathcal{S} \ \text{and } t \in \mathcal{T}$. Consider the set,$$\mathcal{A}=\{as+bt | a,B \in \mathcal{F}\}$$
    There are two case.
    \begin{enumerate}
      \item $\exists v \in \mathcal{V}$ and $v \in \mathcal{S}$. Now let $B=0$, we get $$\mathcal{A}=\{as | a \in \mathcal{F}\}$$
      In this case, we can know that $\mathcal{A}=\mathcal{S}$
      \item $\exists v_{1},v_{2} \in \mathcal{V}$ and $v_{1},v_{2} \notin \mathcal{S}$. Let $v_{1}=a_{1}s+b_{1}t$
      and $v_{2}=a_{2}s+b_{2}t$ and $b_{i}\ne 0,i=1,2$
      \begin{align*}
        m_{1}v_{1}+m_{2}v_{2}&=m_{1}(a_{1}s+b_{1}t)+m_{2}(a_{2}s+b_{2}t)\\
        &=(m_{1}a_{1}+m_{2}a_{2})s+(m_{1}b_{1}+m_{2}b_{2})t \\
        &=ms+nt
      \end{align*}
      Where is $m,n,m_{i} \in \mathcal{F}$ ,$(i=1,2)$ so $m_{1}v_{1}+m_{2}v_{2}\in \mathcal{A}$
    \end{enumerate}
    Let subspace $\mathcal{T}=\{v\in \mathcal{V}| v\notin \mathcal{S},v\ne 0\}$. We can say that set $\mathcal{A}$ can cover
    whole space of $\mathcal{V}$.\\
    We have $$\mathcal{V}=\mathcal{S}+\mathcal{T}$$
    And so anything subspace $\mathcal{S}$ has a complement in $\mathcal{V}$ which called $\mathcal{T}$ and
    $\mathcal{T}=\{\forall v \in \mathcal{V} | v \notin \mathcal{S},v\ne 0 \}$.\\
    Done.

  \end{proof}

  \item[Problem 2](4/10, Behavior of basis over direct sums):{

  Let $\mathcal{V}$ be a vector space.
  \begin{enumerate}
    \item Let $\mathfrak{B}$ be a basis for $\mathcal{\mathcal{V}}$. Suppose that  there
    exist subsets $\mathfrak{B}_{1},\mathfrak{B}_2$ of $\mathfrak{B}$,
    such that $\mathfrak{B}=(\mathfrak{B}_{1}\cup\mathfrak{B}_{2})$ and
     $\mathfrak{B}_{1}\cap\mathfrak{B}_{2}=\emptyset$. Then show
     that $\mathcal{\mathcal{V}}=span(\mathfrak{B}_{1})\oplus span(\mathfrak{B}_{2})$.
      \item Let $\mathcal{\mathcal{V}}=\mathcal{S}\oplus\mathcal{T}$, and let $\mathfrak{B}_{1}$ be a basis for $\mathcal{S}$ and $\mathfrak{B}_{2}$ be a basis for $\mathcal{T}$.
    Show that $\mathfrak{B}_{1}\cap\mathfrak{B}_{2}$ and that $\mathfrak{B}_{1}\cup\mathfrak{B}_{2}$ is a basis for $\mathcal{V}$.
  \end{enumerate}
  }
  \begin{proof}
    There are 2 questions.
    \begin{enumerate}
      \item Let $\mathfrak{B}=\{b_{i} | i=1,2,...,n\}$, and
      $\mathfrak{B}_{1}=\{b_{i} | i=1,2,...,k\  k<n\}$,$\mathfrak{B}_{2}=\{b_{i}| i=k+1,k+2,...,n\}$
      which satisfies
      $\mathfrak{B}=(\mathfrak{B}_{1}\cup\mathfrak{B}_{2})$ and
       $\mathfrak{B}_{1}\cap\mathfrak{B}_{2}=\emptyset$.
       $$\forall  v  \in   \mathcal{V}$$
       We have that,
         $$v=a_{1}b_{1}+a_{2}b_{2}+...+a_{n}b_{n}$$
        and
         $$v_{1}=a_{1}b_{1}+a_{2}b_{2}+...+a_{k}b_{k} \in \text{span($\mathfrak{B}_{1}$)}$$
         $$v_{2}=a_{k+11}b_{k+1}+a_{k+2}b_{K+2}+...+a_{n}b_{n} \in \text{span($\mathfrak{B}_{2}$)}$$
       We get that $v=v_{1}+v_{2}$.So, $$\mathcal{V}=\mathfrak{B}_{1}+\mathfrak{B}_{2}$$
       To prove independence of $\mathfrak{B}_{1}$ and $\mathfrak{B}_{2}$,
       by contrary, suppose there exists $u_{1}\in \text{span($\mathfrak{B}_{1}$)}$ and
       $u_{2}\in \text{span($\mathfrak{B}_{2}$)}$, satisfying that
        $$r_{1}u_{1}+r_{2}u_{2}=0\quad r_{1},r_{2}\in F$$
        And $r_{i}$ aren't all $0$ . Then we write as
        \begin{align*}
          u_{1}&=a_{1}b_{1}+a_{2}b_{2}+...+a_{k}b_{k}\\
          u_{2}&=a_{k+1}b_{k+1}+a_{k+2}b_{k+2}+...+a_{n}b_{n}\\
          r_{1}(a_{1}b_{1} +a_{2}b_{2} &  +... +a_{k}b_{k})+ r_{2}(a_{k+1}b_{k+1}+...+a_{n}b_{n})=0\\
          \to\quad  m_{1}b_{1}+m_{2}b_{2} & +...+m_{k}b_{k}+m_{k+1}b_{k+1}+...+m_{n}b_{n}=0
        \end{align*}
        This is contradicting the independence of basis. So
        $$\text{$\mathfrak{B}_{1}\cap\mathfrak{B}_{2}=\{0\}$}$$
        \item Let $\mathfrak{B}_{1}=\{s_{i} | i \in I_{1}\}$
        and $\mathfrak{B}_{2}=\{t_{j} | j \in I_{2}\}$. We have that,
        $$\forall v \in \mathcal{V}\ \  \exists s \in \mathcal{S} \ \  \exists t \in \mathcal{T} \ \ v=s+t$$
        \begin{align*}
          \forall s \in \mathcal{S} \to s &=\sum_{i} a_{i}s_{i} \quad i \in I_{1}\\
          \forall t \in \mathcal{T} \to t &=\sum_{j} b_{j}t_{j} \quad j \in I_{2}
        \end{align*}
        So we can say that $\forall v \in \mathcal{V}$  , $v=a_{i}s_{i}+b_{j}t_{j}$.So
         $\mathfrak{B}_{1}\cap \mathfrak{B}_{2}$ is the spanning set for
        $\mathcal{V}$\\
        Supose that, there exists a set of nonzero coefficents $a_{i}$ and $b_{j}$ , let
        $$\sum_{i} a_{i}s_{i}+\sum_{j} b_{j}t_{j}=0$$
        Let $s=\sum_{i} a_{i}s_{i}$ and $t=\sum_{j} b_{j}t_{j}$, then we get $s=t$
        which is contradicting $\mathcal{S}\cap\mathcal{T}=\{0\}$. So $\mathfrak{B}_{1}$ and $\mathfrak{B}_{2}$ are linearly independent.
        So $\mathfrak{B}_{1}\cap\mathfrak{B}_{2}$ and that $\mathfrak{B}_{1}\cup\mathfrak{B}_{2}$ is a basis for $\mathcal{V}$.
    \end{enumerate}
      Done.
    \end{proof}
    \item[Problem 3](3/10, Characterization of a basis):{

    Prove Theorem 1.7 in Roman by proving that 1)$\Rightarrow$ 4)$\Rightarrow$  3)$\Rightarrow$  2)$\Rightarrow$  1).\\
    ( In class, I proved it by showing that 1)$\Rightarrow$  2)$\Rightarrow$  3)$\Rightarrow$  4)$\Rightarrow$  1)).
    }
    \begin{proof}
      We do four steps to finish this problem.
      \begin{enumerate}
        \item 1)$\Rightarrow $ 4)\\
        Suppose that 1) holds and $\mathcal{S}$ isn't the maximal linearly independent set,
        let $\exists k \in \mathcal{V}$ is linearly independent to $\mathcal{S}$.
        So we cannot find $k$ be a linear combination of $\mathcal{S}$ which is constradicting the fact that $\mathcal{S}$
        is a spanning set of $\mathcal{V}$.\\
        Hence 1) implies 4).
        \item 4)$\Rightarrow$ 3)\\
        Suppose that 4) holds and $\mathcal{S}$ isn't the minimal spanning set, and let $\mathcal{S}-\mathcal{S}'$ be
        the minimal spanning set. But we know that $\mathcal{S}$ is the maximal linearly independent set.
        we cannot write as $$\forall s' \in \mathcal{S}'\quad s'=\sum_{i} a_{i}s_{i} \quad s_{i} \in \mathcal{S},a_{i} \in \mathcal{F}$$
        Every element of $\mathcal{S}'$ cannot be a linear combination of the elements of $\mathcal{S}$,
        contradicting the assume. Hence 4) implies 3).
        \item 3)$\Rightarrow$ 2)\\
        Suppose that 3) holds and
        \begin{equation*}
          0\ne v=a_{1}s_{1}+...+a_{n}s_{n}=b_{1}s_{1}+...+b_{n}s_{n}
        \end{equation*}
        Where the $a_{i}\ne b_{i}$. By subtracting, we can get that,
        \begin{equation*}
          (a_{1}-b_{1})s_{1}+...+(a_{n}-b_{n})s_{n}=0
        \end{equation*}
        So $\mathcal{S}$ isn't the linearly independent set. It means that
        $$\exists s \in \mathcal{S} \quad s=\sum_{i\ne j} a_{i}s_{i}$$
        It constradicts that $\mathcal{S}$ is the minimal spanning set.Hence 3) implies 2).
        \item 2)$\Rightarrow$ 1)\\
        Supposed that 2) hold and that $\mathcal{S}$ is not linearly independent or soans $\mathcal{V}$.
        Not linearly independence means $0$ is a linear combination of vectors in $\mathcal{S}$.
        And not spans $\mathcal{V}$ totally contradicts that every nonzero vector $v\in \mathcal{V}$ is a linear combination of vectors in $\mathcal{S}$.
        Hence 2) $\Rightarrow$ 1).
      \end{enumerate}
      Done.
    \end{proof}
\end{description}
\end{document}
