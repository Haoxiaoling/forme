\documentclass[a4paper]{article}
\usepackage{xeCJK}
\usepackage{geometry}
\geometry{left=2.5cm,right=2.5cm,top=3cm,bottom=3cm}
\title{Matrix Analysis Homework 7}
\author{龙肖灵 \\Xiaoling Long\\Student ID.:81943968\\email:longxl@shanghaitech.edu.cn}
\usepackage{graphicx}
\usepackage[colorlinks,linkcolor=red]{hyperref}
\usepackage{amsmath, amsthm, amssymb}
\usepackage{mathtools}
\usepackage{subfloat}
\usepackage{stmaryrd}
\usepackage{enumerate}
\newtheorem{prop}{Proposition}
\usepackage{ulem}
\usepackage{indentfirst}
\begin{document}
\maketitle

\begin{description}
  \item[Problem 1]:\\
  Let $A$,$B$ be positive-semidefinite matrices. Show that all eigenvalues of $AB$ are non-negative.
   Is it true that $AB$ is positive-semidefinite? Justify your answer.

  \begin{proof}\
    \begin{enumerate}[i)]
      \item
      Pick an eigenpair $(x,\lambda)$ of $AB$. We have that
      \begin{align*}
        ABx&=\lambda x\\
        x^{T}B^{T}(ABx)&=\lambda x^{T}B^{T}x\\
        (Bx)^{T}A(Bx)&=\lambda x^{T}Bx\\
        (x^{\prime})^{T}Ax^{\prime}&=\lambda x^{T}Bx
      \end{align*}
      And we know that $A$ and $B$ are positive-semidefinite matrix, for all $x\in R^{n}$ and $x^{\prime}\in R^{n}$, such that $x^{T}Bx\ge 0$ and $(x^{\prime})^{T}Ax^{\prime}\ge 0$. So $\lambda x^{T}Bx \ge 0\Rightarrow \lambda\ge 0$.
      So, all eigenvalues of $AB$ are non-negative.
      \item Not always true that $AB$ is positive-semidefinite. $(AB)^{T}=B^{T}A^{T}=BA\ne AB$. Since there isn't enought condition to ensure that $AB$ is a symmetric matrix.
    \end{enumerate}
    Done.
  \end{proof}

  \item[Problem 2]:\\
   Let $A$,$B$ be positive-semidefinite matrices. Show that $\interleave A-B\interleave _{2}\le \max\{\interleave A\interleave_{2},\interleave B\interleave_{2}\}$.

  \begin{proof}\ \\
    Let $C$ be a positive-semidefinite matrix, pick an eigenpair $(x,\lambda)$ of $C$. Then,
    \begin{align*}
      Cx&=\lambda x\\
      C^{T}Cx&=\lambda Cx\\
      C^{T}Cx&=\lambda \lambda x\\
      C^{T}Cx&=\lambda^{2}x
    \end{align*}
    So the eigenvalues of $C^{T}C$ equal the eigenvalues of $C$ power $2$. And all eigenvalues of $C$ are non-negative, so $\|C\|_{2}=\sqrt{\lambda_{\max}(C^{T}C)}=\max{|\lambda(C)|}$.
    Since $(A-B)^{T}=A-B$, so there are two cases
    \begin{enumerate}[i)]
      \item $|\lambda_{\max}(A-B)|\ge |\lambda_{\min}(A-B)|$: $$\interleave A-B\interleave_{2}=\max\limits_{\|x\|_{2}=1}x^{T}(A-B)x=\max\limits_{\|x\|_{2}}(x^{T}Ax-x^{T}Bx)\le \max\limits_{\|x\|_{2}}x^{T}Ax=\lambda_{\max}(A)=\interleave A\interleave_{2}$$
      \item $|\lambda_{\max}(A-B)|< |\lambda_{\min}(A-B)|$:$$\interleave A-B\interleave_{2}=\max\limits_{\|x\|_{2}=1}x^{T}(B-A)x=\max\limits_{\|x\|_{2}}(x^{T}Bx-x^{T}Ax)\le \max\limits_{\|x\|_{2}}x^{T}Bx=\lambda_{\max}(B)=\interleave B\interleave_{2}$$
    \end{enumerate}
    Finally, we can say that $$\interleave A-B\interleave _{2}\le \max\{\interleave A\interleave_{2},\interleave B\interleave_{2}\}$$
    Done.
  \end{proof}

  \item[Problem 3]:\\
  Show that the set of all positive-semidefinite matrices of size $n\times n$ is a convex cone of $\mathbb{R}^{n\times n}$ (you need to read up the definition of "convex cone").

  \begin{proof}
    \textit{Terminology: A cone $C$ is a convex cone if $\alpha x + \beta y$ belongs to $C$, for any positive scalars $\alpha$, $\beta$, and any $x$, $y$ in $C$.}\\
    For all $A$ and $B$ in the set of all positive-semidefinite matrices $\mathcal{C}$ of size $n\times n$. We have that for all $x\in R^{n}$ , such that $x^{T}Ax\ge 0$ and $x^{T}Bx\ge 0$.\\
    For any positive scalars $\alpha$ and $\beta$.
    \begin{align*}
      x^{T}Ax\ge 0 \quad & x^{T}Bx\ge 0\\
      \alpha x^{T}Ax\ge 0 \quad & \beta x^{T}Bx\ge 0\\
      x^{T}\alpha A x\ge 0 \quad & x^{T}\beta Bx\ge 0\\
      x^{T}\alpha Ax+x^{T}\beta Bx &\ge 0\\
      x^{T}(\alpha A+\beta B)x&\ge 0
    \end{align*}
    It means that for all $x\in R^{n}$for any positive scalars $\alpha$ and $\beta$, for any $A$, $B$ in $\mathcal{C}$, $x^{T}(\alpha A+\beta B)x\ge 0 $ alwways remains. So $\alpha A +\beta B$ belongs to $\mathcal{C}$.\\
    So $\mathcal{C}$ is a convex cone of $R^{n\times n}$.\\
    Done.
  \end{proof}

  \item[Problem 4]:\\
  Let $A\in \mathbb{R}^{n\times n}$ be positive-semidefinite. Let $B\in \mathbb{R}^{n\times k}$ be any matrix. Show that the matrix $B^{T}AB$ is positive-semidefinite.

  \begin{proof}\ \\
    Since $A\ge 0$, so we can say that $\exists C \in R_{n\times n}$ such that $A=C^{T}C$. So we can do these as following,
    \begin{align*}
      x^{T}B^{T}ABx&=x^{T}B^{T}C^{T}CBx\\
      &=(CBx)^{T}CBx\\
      &=\|CBx\|_{2} \\
      &\ge 0
    \end{align*}
    Finally, we get that $x^{T}B^{TAB}x\ge 0$. So the matrix $B^{T}AB$ is positive-simidefinite.
  \end{proof}

  \item[Problem 5]:\\
   Let $A\in \mathbb{R}^{n\times n}$ be positive-semidefinite. Show that for every distinct $i\ne j$ we have that $a_{ii}a_{jj}\ge a_{ij}^{2}$.

  \begin{proof}\ \\
    Since $A\ge 0$, so we know that every principal minor of $A$ is greater than or equal to $0$. So When we remove all columns and rows whose indices not equal to $i$ and $j$. The principal will be
    $A(i,j)=\begin{pmatrix}
      a_{ii} & a_{ij}\\
      a_{ji} & a_{jj}
    \end{pmatrix}$. And $det(A(i,j))\ge 0$, then we can get that $a_{ii}a_{jj}-a_{ij}a_{ji}\ne 0$. That $A\ge 0$ implies that $A=A^{T}$ which means that $a_{ij}=a_{ji}$. Finally we can get that $$a_{ii}a_{jj}\ge a_{ij}^{2}$$
    Done.
  \end{proof}

  \item[Problem 6]:\\
    Let $A\in \mathbb{R}^{n\times n}$ be symmetric, strictly diagonally dominant, and suppose that $a_{ii}>0$. Prove that $A$ is positive-definite.

  \begin{proof}\ \\
      From \textit{Gerschgorin Circles theorem}, all $\lambda_{i} \in \sigma{A}$ are in the $i-th$ Gerschgorin Circle defined by $$|z-a_{ii}|\le \sum\limits_{j=1\atop j\ne i}^{n}|a_{ij}|$$
      And $A$ is strict diagonally dominant. It means that $a_{ii}> \sum\limits_{j=1\atop j\ne i}^{n}|a_{ij}|$ and $a_{ii}>0$.
      \begin{align*}
      & \qquad\qquad  |z-a_{ii}|\le \sum\limits_{j=1\atop j\ne i}^{n}|a_{ij}|\\
      &\Rightarrow  -\sum\limits_{j=1\atop j\ne i}^{n}|a_{ij}|<z-a_{ii}<\sum\limits_{j=1\atop j\ne i}^{n}|a_{ij}|\\
      &\Rightarrow  a_{ii}-\sum\limits_{j=1\atop j\ne i}^{n}|a_{ij}|<z<a_{ii}+\sum\limits_{j=1\atop j\ne i}^{n}|a_{ij}|
      \end{align*}
      It implies that all elements in the circle are greater than $0$. It also implies that $\lambda_{i}>0$. \\
      And we also can get these
      \begin{align*}
        Ax&=\lambda x\\
        x^{T}Ax&=\lambda x^{T}x\\
        &=\lambda \|x\|_{2}
      \end{align*}
      For all $x^{T}x\ne 0$, we always get that $x^{T}Ax=\lambda \|x\|_{2}>0$, and $A$ also is symmetric matrix.\\
       So $A$ is positive-definite.\\
       Done.
  \end{proof}

\end{description}

\end{document}
